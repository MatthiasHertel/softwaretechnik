%%%%%%%%%%%%%%%%%%%%%%%%%%%%%%%%%%%%%%%%%%%%%%%%%%%%%%%%%%%%%%%%%%%%%
% LaTeX Template: Softwaretechnik SS 2017
%
% Date: April 2017
%
%%%%%%%%%%%%%%%%%%%%%%%%%%%%%%%%%%%%%%%%%%%%%%%%%%%%%%%%%%%%%%%%%%%%%%

\documentclass[12pt]{article}
\usepackage[a4paper]{geometry}
\usepackage{framed}
\usepackage[myheadings]{fullpage}
\usepackage{fancyhdr}
\usepackage{lastpage}
\usepackage{graphicx, wrapfig, subcaption, setspace, booktabs}


\usepackage[font=small, labelfont=bf]{caption}
\usepackage[protrusion=true, expansion=true]{microtype}
\usepackage[ngerman]{babel}
\usepackage[ngerman]{translator}
\usepackage{sectsty}
\usepackage{url, lipsum}
\usepackage[parfill]{parskip}
\usepackage{csquotes}
\usepackage[hidelinks]{hyperref}
\usepackage[acronym]{glossaries}

\usepackage[sorting=none,backref=true, backend=biber]{biblatex}
\addbibresource{\jobname.bib}
\usepackage[export]{adjustbox}
\usepackage{multicol}
\usepackage{tikz}
\usepackage{float}
\usepackage{pdfpages}
% \usepackage{xcolor}


\makeglossaries
\glstoctrue

\usepackage{listings}
\usepackage{color}

\usepackage{dirtree}
\input dirtree

\definecolor{dkgreen}{rgb}{0,0.6,0}
\definecolor{gray}{rgb}{0.5,0.5,0.5}
\definecolor{mauve}{rgb}{0.58,0,0.82}
\definecolor{dandelion}{HTML}{FDBC42}

\lstset{
  % frame=tblr,
  language=Java,
  aboveskip=3mm,
  belowskip=3mm,
  showstringspaces=false,
  columns=flexible,
  basicstyle={\small\ttfamily},
  numbers=left,
  numberstyle=\tiny\color{gray},
  keywordstyle=\color{blue},
  commentstyle=\color{dkgreen},
  % morecomment=[s][\color{orange}]{@}{\ },
  stringstyle=\color{mauve},
  breaklines=true,
  breakatwhitespace=true,
  tabsize=3,
  moredelim=[is][\textcolor{orange}]{\~\~}{\~\~},
  moredelim=[il][\textcolor{dandelion}]{\$\$},
  moredelim=[is][\textcolor{dandelion}]{\%\%}{\%\%}
}

\DTsetlength{ 0.2em}{ 1em}{ 0.2em}{ 0.4pt}{ 0.1pt}

\usepackage[T1]{fontenc}
%-------------------------------------------------------------------------------
% Commands
%-------------------------------------------------------------------------------
\newcommand{\HRule}[1]{\rule{\linewidth}{#1}}
\input{../env}
%-------------------------------------------------------------------------------
% HEADER & FOOTER
%-------------------------------------------------------------------------------
\fancypagestyle{myplain}
{
\fancyhf{}
\paperwidth=\pdfpageheight
\paperheight=\pdfpagewidth
\pdfpageheight=\paperheight
\pdfpagewidth=\paperwidth
\headwidth=\textwidth
\renewcommand\headrulewidth{0pt}
\renewcommand\footrulewidth{0pt}
\fancyfoot[R]{Seite \thepage\ von \pageref{LastPage}}

}

\fancypagestyle{myfancy}{
  \fancyhf{}
  \pagestyle{fancy}
  \fancyhf{}
  \setlength\headheight{15pt}
  \fancyhead[L]{\newCommandName}
  \fancyhead[R]{\newCommandUniversity}
  \fancyfoot[R]{Seite \thepage\ von \pageref{LastPage}}
}

%-------------------------------------------------------------------------------
% TITLE PAGE
%-------------------------------------------------------------------------------
\begin{document}
\hypersetup{
    % colorlinks,
    citecolor=black,
    filecolor=black,
    linkcolor=black,
    urlcolor=black
}


\title{ \normalsize
		\HRule{0.5pt} \\
		\LARGE \textbf{\uppercase{\newCommandDiscipline}} \\
    \smallbreak
    \small\textbf{{TST - Objektorientiertes Testen und Test Driven Development}}\\
		\HRule{2pt} \\ [0.5cm]
    \small\textbf{{\newCommandTerm}}\\
    [0.5cm]
    \normalsize \today \vspace*{10\baselineskip}}

\date{}



\author{
    \newCommandName \\
		\newCommandMatriculationNumber \\
		\newCommandUniversity \\
		\newCommandFaculty
}

% \pagenumbering{gobble}

\maketitle
\thispagestyle{empty}

\newpage
\pagestyle{myfancy}
\tableofcontents
% \listoffigures
\newpage

%-------------------------------------------------------------------------------
% Section title formatting
\sectionfont{\scshape}
%-------------------------------------------------------------------------------

%-------------------------------------------------------------------------------
% BODY
%-------------------------------------------------------------------------------
\section{TST - Objektorientiertes Testen und Test Driven Development}
\subsection{Vorwort}
\subsubsection{Erkl"arung zur Aufgabenumsetzung}
Getest wird die Klasse \textbf{ProductController.java}.
Sie repr"asentiert einen Crud-Controller. Es werden die Crud-Operationen getestet. Es wird auf eine Exception getestet. (Wenn die Resource unter der Id - ProductNotFoundException - nicht gefunden wurde.)


\subsubsection{Git Repository}
\textbf{SSH:}
% \begin{lstlisting}[]

git clone git@github.com:MatthiasHertel/productstore.git
% \end{lstlisting}

\textbf{Download:}

https://github.com/MatthiasHertel/productstore/archive/master.zip

\newpage
\subsection{Project - Tree}
\renewcommand*\DTstylecomment{\rmfamily\color{red}\textsc}
\dirtree{%
.1 / .
.2 src .
  .3 main .
    .4 java .
      .5 de .
        .6 mhertel .
          .7 ProductstoreApplication.java.
          .7 config .
            .8 SecurityConfig.java.
          .7 controller .
            .8 HomeController.java.
            .8 ProductController.java\DTcomment{Class}.
          .7 domain .
            .8 Product.java.
            .8 User.java.
            .8 Security .
              .9 Authority.java.
              .9 PasswordResetToken.java.
              .9 Role.java.
              .9 UserRole.java.
          .7 repository .
            .8 ProductRepository.java.
            .8 RoleRepository.java.
            .8 UserRepository.java.
          .7 service .
            .8 ProductService.java.
            .8 UserService.java.
            .8 impl .
              .9 ProductServiceImpl.java.
              .9 UserSecurityService.java.
              .9 UserServiceImpl.java.
          .7 utility .
            .8 ProductNotFoundException.java\DTcomment{Exception}.
            .8 SecurityUtility.java.
    .4 resources .
  .3 test.
    .4 java .
      .5 de .
        .6 mhertel .
          .7 controller .
            .8 ProductControllerTest.java\DTcomment{Testclass}.
          .7 ProductstoreApplicationTests.java.
}
%-------------------------------------------------------------------------------
% #1
%-------------------------------------------------------------------------------
\newpage
\subsection{ProductController.java}

\begin{lstlisting}
package de.mhertel.controller;

import de.mhertel.domain.Product;
import de.mhertel.service.ProductService;
import de.mhertel.utility.ProductNotFoundException;
import org.springframework.beans.factory.annotation.Autowired;
import org.springframework.stereotype.Controller;
import org.springframework.ui.ModelMap;
import org.springframework.validation.BindingResult;
import org.springframework.web.bind.annotation.ModelAttribute;
import org.springframework.web.bind.annotation.RequestMapping;
import org.springframework.web.bind.annotation.RequestMethod;
import org.springframework.web.bind.annotation.RequestParam;
import java.util.List;


/**
 * Created by matthias on 13.05.17.
 */

$$@Controller
%%@RequestMapping%%("/products")
public class ProductController {


    $$@Autowired
    private ProductService productService;


    /**
     * This method will list all existing products.
     *
     * @param model
     * @return route to View with all Products TODO pagination
     */
    %%@RequestMapping%%(value = "/list" , method = RequestMethod.GET)
    public String ~~listProducts~~(ModelMap model) {

        List<Product> productList = productService.findAll();
        model.addAttribute("productList", productList);

        return ("product/list");
    }





    /**
     * This method will provide the medium to add a new product.
     *
     * @param model
     * @return route to View with Form for Product TODO DRY - use one form for create/update
     */
    %%@RequestMapping%%(value = "/add", method = RequestMethod.GET)
    public String ~~addProduct~~(ModelMap model) {

        Product product = new Product();
        model.addAttribute("product", product);
        model.addAttribute("edit", false);
        return "product/add";
    }



    /**
     * This method will be called on form submission, handling POST request for saving product in database.
     * TODO validate the user input
     * @param product
     * @param result
     * @param model
     * @return redirect to View with all Products
     */
    %%@RequestMapping%%(value = "/add", method = RequestMethod.POST)
    public String ~~addProductPost~~(%%@ModelAttribute%%("product") Product product, BindingResult result, ModelMap model) {
        if (result.hasErrors()) {
            return "product/list";
        }
        productService.save(product);
        model.addAttribute("success", "Product " + product.getTitle() + " created successfully");
        return "redirect:/products/list";
    }








    /**
     * @param id
     * @param model
     * @return route to View with the product (Single)
     * @throws Exception
     */
    %%@RequestMapping%%("/detail")
    public String ~~productDetail~~(%%@RequestParam%%("id") Long id, ModelMap model) throws Exception{

        Product product = productService.findOne(new Long(id));
        if (product == null) throw new ProductNotFoundException();
        model.addAttribute("product", product);

        return "product/detail";
    }




    /**
     * This method will provide the medium to update an existing product.
     *
     * @param id
     * @param model
     * @return route to View with Form for Product TODO DRY - use one form for create/update
     */
    %%@RequestMapping%%("/update")
    public String ~~updateProduct~~(%%@RequestParam%%("id") Long id, ModelMap model) {

        Product product = productService.findOne(id);
        model.addAttribute("product", product);
        model.addAttribute("edit", true);
        return "product/update";
    }










    /**
     * This method will be called on form submission, handling POST request for updating product in database.
     *
     * @param product
     * @param model
     * @return redirect to View the Product (Single)
     */
    %%@RequestMapping%%(value="/update", method=RequestMethod.POST)
    public String ~~updateProductPost~~(%%@ModelAttribute%%("product") Product product, ModelMap model) {

        productService.updateProduct(product);
        model.addAttribute("success", "Product " + product.getTitle()	+ " updated successfully");

        return "redirect:/products/detail?id="+ product.getId();
    }


    /**
     * This method will delete a Product by it's Id value.
     *
     * @param id
     * @param model
     * @return redirect to View with all Products
     */
    %%@RequestMapping%%(value="/remove", method=RequestMethod.POST)
    public String ~~removeProduct~~(%%@ModelAttribute%%("id") String id, ModelMap model) {

        productService.removeOne(Long.parseLong(id));
        List<Product> productList = productService.findAll();
        model.addAttribute("productList", productList);

        return "redirect:/products/list";
    }
}
\end{lstlisting}
\newpage

\subsection{ProductControllerTest.java}

\begin{lstlisting}
package de.mhertel.controller;

import de.mhertel.domain.Product;
import de.mhertel.service.ProductService;
import de.mhertel.utility.ProductNotFoundException;
import org.junit.Assert;
import org.junit.Before;
import org.junit.Test;
import org.mockito.InjectMocks;
import org.mockito.Mock;
import org.mockito.MockitoAnnotations;
import org.mockito.Spy;
import org.springframework.mock.web.MockHttpServletRequest;
import org.springframework.ui.ModelMap;
import org.springframework.validation.BindingResult;
import java.util.ArrayList;
import java.util.List;
import static org.mockito.Matchers.anyLong;
import static org.mockito.Mockito.*;

/**
 * Created by matthias on 13.05.17.
 */
public class ProductControllerTest {

    $$@Mock
    ProductService productService;
    $$@Mock
    BindingResult result;
    $$@Mock
    MockHttpServletRequest request = new MockHttpServletRequest();
    $$@InjectMocks
    ProductController productController;
    $$@Spy
    List<Product> products = new ArrayList<>();
    $$@Spy
    ModelMap model;

    $$@Before
    public void ~~setUp()~~ throws Exception {
        MockitoAnnotations.initMocks(this);
        products = getProductList();
    }

    /**
     * Seeding some hardcode-Data for testsuite.
     *
     * @return List<Product>
     */
    public List<Product> ~~getProductList()~~{
        Product p1 = new Product();
        p1.setId(new Long(1));
        p1.setTitle("Title1");
        p1.setCategory("Category1");
        p1.setDescription("description");
        p1.setShippingWeight(20);
        p1.setListPrice(20.0);
        p1.setActive(true);
        p1.setInStockNumber(20);

        Product p2 = new Product();
        p2.setId(new Long(2));
        p2.setTitle("Title2");
        p2.setCategory("Category2");
        p2.setDescription("description");
        p2.setShippingWeight(20);
        p2.setListPrice(20.0);
        p2.setActive(true);
        p2.setInStockNumber(20);

        Product p33 = new Product();
        p33.setId(new Long(33));
        p33.setTitle("Title33");
        p33.setCategory("Category33");
        p33.setDescription("description");
        p33.setShippingWeight(330);
        p33.setListPrice(330.0);
        p33.setActive(true);
        p33.setInStockNumber(330);

        products.add(p1);
        products.add(p2);
        products.add(p33);
        return products;
    }





    /**
     * Test method - list all existing products.
     * @throws Exception
     */
    $$@Test
    public void ~~listProducts()~~ throws Exception {
        when(productService.findAll()).thenReturn(products);
        Assert.assertEquals(productController.listProducts(model), "product/list");
        Assert.assertEquals(model.get("productList").toString(), products.toString());
        verify(productService, atLeastOnce()).findAll();
    }

    /**
     * Test method - get addProduct-View-Form.
     * @throws Exception
     */
    $$@Test
    public void ~~addProduct()~~ throws Exception {
        Assert.assertEquals(productController.addProduct(model), "product/add");
        Assert.assertNotNull(model.get("product"));
        Assert.assertFalse((Boolean)model.get("edit"));
        Assert.assertEquals(((Product)model.get("product")).getId(), null);
    }

    /**
     * Test method - post addProduct-View-Form.
     * @throws Exception
     */
    $$@Test
    public void ~~addProductPost()~~ throws Exception {
        when(result.hasErrors()).thenReturn(false);
        doNothing().when(productService).saveProduct(any(Product.class));
        Assert.assertEquals(productController.addProductPost(products.get(0), result, model), "redirect:/products/list");
        Assert.assertEquals(model.get("success"), "Product Title1 created successfully");
    }







    /**
     * Test method - get expected Exception (product not found)
     * @throws Exception
     */
    %%@Test%%(expected = ProductNotFoundException.class)
    public void ~~productDetail()~~ throws Exception {
        productController.productDetail(new Long(34), model);
    }

    /**
     * Test method - get editProduct-View-Form.
     * @throws Exception
     */
    $$@Test
    public void ~~updateProduct()~~ throws Exception {
        Product product = products.get(0);
        when(productService.findOne(anyLong())).thenReturn(product);
        Assert.assertEquals(productController.updateProduct(anyLong(), model), "product/update");
        Assert.assertNotNull(model.get("product"));
        Assert.assertTrue((Boolean)model.get("edit"));
        Assert.assertEquals(((Product)model.get("product")).getId(), new Long(1));
    }

    /**
     * Test method - post editProduct-View-Form.
     * @throws Exception
     */
    $$@Test
    public void ~~updateProductPost()~~ throws Exception {
        when(result.hasErrors()).thenReturn(false);
        doNothing().when(productService).updateProduct(any(Product.class));
        Assert.assertEquals(productController.updateProductPost(products.get(0), model), "redirect:/products/detail?id=1");
        Assert.assertEquals(model.get("success"), "Product Title1 updated successfully");
    }








    /**
     * Test method - removeProduct
     * @throws Exception
     */
    $$@Test
    public void ~~removeProduct~~() throws Exception {
        doNothing().when(productService).removeOne(anyLong());
        Assert.assertEquals(productController.removeProduct("1", model), "redirect:/products/list");
    }
}
\end{lstlisting}

\newpage

\subsection{ProductNotFoundException.java}

\begin{lstlisting}
package de.mhertel.utility;

import org.springframework.http.HttpStatus;
import org.springframework.web.bind.annotation.ResponseStatus;

/**
 * Created by matthias on 13.05.17.
 */
%%@ResponseStatus%%(value=HttpStatus.NOT_FOUND, reason="No such Product")  //404
public class ProductNotFoundException extends RuntimeException {
    // ...
}
\end{lstlisting}




%-------------------------------------------------------------------------------
% Literatur - Glossar - Akronyme
%-------------------------------------------------------------------------------

% \clearpage
% \setlength\bibitemsep{10pt}
% \printbibliography[heading=bibintoc]
% \newpage
% \printglossary[type=main,title=Glossar]
% \printglossary[type=\acronymtype, title=Akronyme]


%-------------------------------------------------------------------------------
% ENDE
%-------------------------------------------------------------------------------

\end{document}
